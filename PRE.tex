\documentclass{article}[12pt]

\usepackage[utf8]{inputenc}
\usepackage[T1]{fontenc}      
\usepackage[english]{babel}
%\usepackage[left=2.00cm, right=2cm, top=2.00cm, bottom=2.00cm]{geometry}
\usepackage{amsthm}
\usepackage{fancyhdr}
\usepackage{amsmath}
\usepackage{amssymb}
\usepackage{mathrsfs}
\usepackage{stmaryrd} 
\renewcommand{\qedsymbol}{}
\usepackage{graphicx}
\graphicspath{{illustrations/}}
\usepackage{caption}
\usepackage{dsfont}
\usepackage{xcolor}
\usepackage{textcomp}

\pagestyle{fancy}
\renewcommand{\headrulewidth}{1pt}
\lhead{}
\chead{\textsc{numerical simulation of sdes with distributional drift}}
\rhead{}

\newcommand{\pade}[2]{\frac{\partial #1}{\partial #2}}
\newcommand{\de}[2]{\frac{\mathrm{d} #1}{\mathrm{d} #2}}
\newcommand{\integ}[4]{\int_{#1}^{#2}{#3\ \mathrm{d} #4}}

\newtheorem{defi}{Definition}[section]
\newtheorem{theo}{Theorem}[section]
\newtheorem{lem}[theo]{Lemma}
\newtheorem{cor}[theo]{Corollary}
\newtheorem{Pro}{Proposition}[section]
\newtheorem{ex}{Example}[section]
\newtheorem{rem}{Remark}[section]

\newcommand{\norme}[1]{\left\Vert #1\right\Vert}
\newcommand{\R}{\mathbb{R}}
\newcommand{\Z}{\mathbb{Z}}
\newcommand{\N}{\mathbb{N}}
\newcommand{\E}{\mathbb{E}}
\newcommand{\D}{D(\Omega)}
\newcommand{\Di}{D'(\Omega)}
\newcommand{\C}{\mathbb{C}}
\newcommand{\Cinf}{\mathcal{C}^\infty(\Omega,\R)}
\newcommand\gts[1]{\og#1\fg}
\newcommand{\n}{n\in\N^*}
\newcommand{\di}{\mathrm{d}}
\newcommand{\loc}{\mathcal{L}_{loc}^{1}(\Omega)}

\begin{document}
\title{\Large \textsc{\textbf{numerical simulation of sdes with distributional drift}}} \author{Maximilien \textsc{Germain} \date{May 2018}}
\maketitle

\section{Introduction}

    \paragraph{}
    We would like to simulate numerically sample paths of the solution of the stochastic differential equation
    \begin{equation} \label{sde}
    \di X_t = b(X_t)\ \di t + \di W_t
    \end{equation}
    where $b\in H^s_q(\R),\ s\in\left]-\frac{1}{2},0\right[$, $t\in[0,T]$, and $W_t$ is a standard Brownian motion. This equation is studied in \cite{Fla-Iss-Rus-2017} in which the authors prove existence and unicity in law of a virtual solution for equation (\ref{sde}). 
    
    \begin{ex}
        An example of such drift $b$ is given by the derivative of a sample path of a fractional Brownian motion $B^H_x$ with Hurst index $1/2<H<1$. These stochastic processes are gaussian processes verifying $$\E\left[B_t^HB_s^H\right]=\frac{1}{2}\left(t^{2H}+s^{2H}+|t-s|^{2H}\right).$$ We note $s = H - 1$. Given $B^H_x(\omega)\in H^{s+1}_q(\R)$, we can take $b(x) = \pade{}{x}B^H_x(\omega)\in H^s_q(\R)$. We will use this in our numerical simulations.
    \end{ex}    
    
    \paragraph{}
    As far as the drift $b$ is not a function but a distribution, it must be approximated if we want to evaluate it at points. In order to do so, we will use a series representation of $b$ and truncate it. That is why we will consider two steps in our algorithm: \begin{enumerate}
        \item approximate the drift $b$ by a function $b^N$ meant to converge to $b$ as $N\rightarrow\infty$.
        \item approximate the solution $X^N_t$ of the approximated SDE
        \begin{equation} \label{sde2}
        \di X^N_t = b^N\left(X^N_t\right)\ \di t + \di W_t
        \end{equation} 
        by $X^{N,n}_t$ defined with the Euler-Maruyama scheme
        \begin{equation*}
        X^{N,n}_t = X_0 + \int_0^t b^N\left(X^{N,n}_{\eta_n(t)}\right)\di t + W_{\eta_n(t)}
        \end{equation*}
        where $\eta_n(t)=t_k$ if $t\in[t_k,t_k+1]$, for $t_k=\frac{k}{2^n}$ with $ k\in\llbracket0,\lceil2^nT\rceil\rrbracket$.
        
    \end{enumerate}

\section{Numerical simulation of fractional Brownian motion}    
    \paragraph{}
    To simulate a sample path of a fractional brownian motion $B^H_x$ on a finite grid $(x_k)_{k\in\llbracket1,n\rrbracket}$, we simulate $n$ independent standard gaussian random variables $(X_k)_{k\in\llbracket1,n\rrbracket}$ and then correlate them with the definite positive correlation matrix 
    $$C_{k,s}=\E\left[B_{x_k}^HB_{x_s}^H\right]=\frac{1}{2}\left(x_k^{2H}+x_s^{2H}+|x_k-x_s|^{2H}\right).$$
    To do so, we use the Cholesky decomposition method and calculate the triangular matrix $M$ such that $C=MM^\top$. Therefore, defining
    $$X = \begin{pmatrix}
    X_1 \\ \vdots \\ X_n
    \end{pmatrix}\ \mathrm{and}\ B^H = MX,$$
    $B^H$ contains the values of a fractional brownian motion evaluated on the grid $(x_k)_{k\in\llbracket1,n\rrbracket}$.
    
\section{Approximation of the drift}
    \subsection{Series representation}
    \paragraph{}
    We use Haar wavelets to give a series representation of $b$. By doing so, we will be able to approximate it numerically by truncating the series.
    
    \begin{defi}[Haar wavelets]
        We define the Haar wavelets $h_{j,m}$ on $\R$ with $j\in\N\cup\{-1\}$ and $m\in\Z$ by:
        $$\begin{cases}
        h_M&:x\longmapsto\left(\mathds{1}_{\left[0,\frac{1}{2}\right[}-\mathds{1}_{\left[\frac{1}{2},1\right[}\right)(x)\\ h_{-1,m}&:x\longmapsto\sqrt{2}|h_M(x-m)|\\
        h_{j,m}&:x\longmapsto h_M(2^jx-m) 
        \end{cases}$$
    \end{defi}
    
    \begin{theo}[See \cite{Iss-Rus-2200}]
        Let $b\in H^s_q(\R)$ for $2\leq q \leq \infty$, and $s\in\left]-\frac{1}{2},\frac{1}{q}\right[$. Therefore,
        \begin{equation}
        b = \sum_{j=-1}^{+\infty}\sum_{m\in\Z}\mu_{j,m}h_{j,m}
        \end{equation}
        where $\mu_{j,m} = 2^j\int_{\R}b(x)h_{j,m}(x)\ \di x$ in the sense of dual pairing.
    \end{theo}

    \begin{defi}
        With the same notation $\mu_{j,m} $, let $b\in H^s_q(\R)$ for $2\leq q \leq \infty$, and $s\in\left]-\frac{1}{2},\frac{1}{q}\right[$. Given $N\in\N^*$ we define $b^N$ by:
        \begin{equation}
        b^N = \sum_{m=N}^{N-1}\mu_{-1,m}h_{-1,m}+\sum_{j=0}^{N}\sum_{m=-N2^j}^{N2^j-1}\mu_{j,m}h_{j,m}.
        \end{equation}
    \end{defi}

    \begin{rem}
        We can note that $\mathrm{Supp}\ b^N\subset [-N,N].$ Moreover, we have: $$\norme{b-b^N}_{H_q^s(\R)} \underset{N\rightarrow+\infty}{\longrightarrow} 0.$$
    \end{rem}

    \subsection{Computation of the coefficients $\mu_{j,m}$ when $b$ is the derivative of a fractional brownian motion}
        Faber basis
\section{Numerical results}

\newpage
\section{Convergence} 
    
    \subsection{Convergence of $X_s^{N,n}$ to $X_s^N$ in $L^2$}
        \paragraph{}
        Recently, Leobacher and Szölgyenyi proved in \cite{Leo-Szo} the convergence of the Euler-Maruyama scheme for SDE with discontinuous but piecewise Lipschitz drift and with a degenerate diffusion coefficient. This framework applies to the scheme we use with piecewise constant drift $b^N$ and a constant diffusion coefficient.
    
    \begin{theo}[Theorem 3.1. in \cite{Leo-Szo}]
        $\forall \varepsilon >0,\ \exists C_N>0,\ \exists n_0\in\N$ such that $\forall n\geq n_0$
        \begin{equation}
        \E\left[\underset{0\leq t\leq T}{\sup}\left|X^{N,n}_t-X^N_t\right|^2\right]^{1/2}\leq C_N\delta^{1/4-\varepsilon}
        \end{equation}
        with $\delta=\frac{1}{2^n}$ the step size.
    \end{theo}

    \textbf{TO DO: make $C_N$ explicit}.
    
    \subsection{Convergence of $X_s^{N}$ to $X_s$}
        \paragraph{}
        We want to estimate $w(y):=\E\left[\underset{0\leq t\leq y}{\sup}\left|X^{N}_t-X_t\right|^2\right]$ for $y\in[0,T]$. In order to do so, we must go back to the definition of the virtual solution of the SDE (\ref{sde}) given in \cite{Fla-Iss-Rus-2017}. The authors define the virtual solution of SDE (\ref{sde}) by:
        \begin{equation}
        X_t = X_0 + u(0,X_0) - u(t,X_t) + (\lambda+1)\int_0^t u(s,Xs)\ \di s +\int_0^t \nabla u(s,Xs)\ \di W_s + W_t
        \end{equation}
        where $u$ is the mild solution in $H_p^{1+\delta}$ of the following parabolic PDE:
        \begin{equation}\label{pde}
        \begin{cases}
        \partial_t u + \frac{1}{2}\Delta u + b\nabla u - (\lambda+1)u = -b\ &\mathrm{on}\ [0,T]\times\R\\
        u(T) = 0\ &\mathrm{on}\ \R
        \end{cases}.
        \end{equation}
        
        We also define another similar PDE by replacing $b$ by $b^N$, and noting $u^N$ its mild solution in $H_p^{1+\delta}$.
        \begin{equation}\label{pde2}
        \begin{cases}
        \partial_t u^N + \frac{1}{2}\Delta u^N + b^N\nabla u^N - (\lambda+1)u^N = -b^N\ &\mathrm{on}\ [0,T]\times\R\\
        u^N(T) = 0\ &\mathrm{on}\ \R
        \end{cases}.
        \end{equation}
        
        \paragraph{}              
        We recall a useful lemma concerning the solutions of (\ref{pde}) and (\ref{pde2}).
        
        \begin{lem}[Lemma 19 in \cite{Fla-Iss-Rus-2017}]\label{lem}
            Let $(\delta,p)\in K(\beta,q)$ and let $v_\lambda$ be the mild solution to (\ref{pde}) in $H_p^{1+\delta}$. Fix $\rho$ such that the integral operator is a contraction and let $\lambda>\rho$. Then $v_\lambda(t)\in\mathcal{C}^{1,\alpha}$ with $\alpha=\delta-1/p$ for each fixed $t$ and 
            \begin{equation}
            \underset{(t,x)\in[0,T]\times\R}{\sup} |\nabla v_\lambda(t,x)| \rightarrow 0,\ as\ \lambda \rightarrow \infty
            \end{equation}
            where the choice of $\lambda$ depends only on $\delta,\beta,\norme{b}_{H_p^{-\beta}}$, and $\norme{b}_{H_q^{-\beta}}$.
        \end{lem}
        
        By lemma \ref{lem}, we choose $\lambda$ big enough for $\nabla u$ and $\nabla u^N$ to be bounded by $\frac{1}{\sqrt{14}}$. $\lambda$ can be chosen independently of $N$ as far as $\norme{b - b^N}_{H_q^s(\R)} \underset{N\rightarrow\infty}{\longrightarrow} 0$ (See Step 2 of the proof of Proposition 26 in \cite{Fla-Iss-Rus-2017}). Therefore $u^N$ and $u$ are $\frac{1}{\sqrt{14}}$-lipschitz.
        
        \begin{multline*}
        \left|X^{N}_t-X_t\right|^2 = \bigg|u^N(0,X_0) - u(0,X_0) + u(t,X_t) - u^N(t,X_t)+ u^N(t,X_t)\\- u^N(t,X_t^N) + (\lambda+1)\int_0^t \{u^N(s,X_s^N)- u^N(s,X_s) + u^N(s,X_s) - u(s,X_s)\}\ \di s \\+ \int_0^t \{\nabla u^N(s,X_s^N) - \nabla u^N(s,X_s) + \nabla u^N(s,X_s) - \nabla u(s,X_s)\}\ \di W_s\bigg|^2
        \end{multline*}
        \begin{multline*}
        \leq 7\Bigg(\Big|u^N(0,X_0) - u(0,X_0)\Big|^2 + \Big|u(t,X_t) - u^N(t,X_t)\Big|^2 + \Big|u^N(t,X_t)- u^N(t,X_t^N)\Big|^2\\ + \bigg|(\lambda+1)\int_0^t \{u^N(s,X_s^N)- u^N(s,X_s)\}\ \di s\bigg|^2 + \bigg|(\lambda+1)\int_0^t \{u^N(s,X_s) - u(s,X_s)\}\ \di s\bigg|^2 \\+ \bigg|\int_0^t \{\nabla u^N(s,X_s^N) - \nabla u^N(s,X_s)\}\ \di W_s\bigg|^2 + \bigg|\int_0^t \{\nabla u^N(s,X_s) - \nabla u(s,X_s)\}\ \di W_s\bigg|^2 \Bigg)
        \end{multline*}
        \begin{multline*}
        \leq 7\Bigg(\Big|u^N(0,X_0) - u(0,X_0)\Big|^2 + \Big|u(t,X_t) - u^N(t,X_t)\Big|^2 + \frac{1}{14}\big|X_t- X_t^N\big|^2\\ + (\lambda+1)\ t\int_0^t \big|u^N(s,X_s^N)- u^N(s,X_s)\big|^2\ \di s + (\lambda+1)\ t\int_0^t \big|u^N(s,X_s) - u(s,X_s)\big|^2\ \di s \\\bigg|\int_0^t \{\nabla u^N(s,X_s^N) - \nabla u^N(s,X_s)\}\ \di W_s\bigg|^2 + \bigg|\int_0^t \{\nabla u^N(s,X_s) - \nabla u(s,X_s)\}\ \di W_s\bigg|^2 \Bigg)
        \end{multline*}
        \begin{multline*}
        \leq 7\Bigg(\Big|u^N(0,X_0) - u(0,X_0)\Big|^2 + \Big|u(t,X_t) - u^N(t,X_t)\Big|^2 + \frac{1}{14}\big|X_t- X_t^N\big|^2\\ + \frac{\lambda+1}{14}\ t\int_0^t \big|X_s^N- X_s\big|^2\ \di s + (\lambda+1)\ t\int_0^t \big|u^N(s,X_s) - u(s,X_s)\big|^2\ \di s \\+ \bigg|\int_0^t \{\nabla u^N(s,X_s^N) - \nabla u^N(s,X_s)\}\ \di W_s\bigg|^2 + \bigg|\int_0^t \{\nabla u^N(s,X_s) - \nabla u(s,X_s)\}\ \di W_s\bigg|^2 \Bigg).
        \end{multline*}
        
        It follows that:
        \begin{multline*}
        \underset{0\leq t\leq y}{\sup}\left|X^{N}_t-X_t\right|^2\leq 14\underset{0\leq t\leq y}{\sup}\Bigg(\Big|u^N(0,X_0) - u(0,X_0)\Big|^2 + \Big|u(t,X_t) - u^N(t,X_t)\Big|^2 \\ + \frac{\lambda+1}{14}\ t\int_0^t \big|X_s^N- X_s\big|^2\ \di s + (\lambda+1)\ t\int_0^t \big|u^N(s,X_s) - u(s,X_s)\big|^2\ \di s \\\bigg|\int_0^t \{\nabla u^N(s,X_s^N) - \nabla u^N(s,X_s)\}\ \di W_s\bigg|^2 + \bigg|\int_0^t \{\nabla u^N(s,X_s) - \nabla u(s,X_s)\}\ \di W_s\bigg|^2 \Bigg)
        \end{multline*}
        \begin{multline*}
        \leq 14\Bigg(\underset{0\leq t\leq y}{\sup}\Big|u^N(0,X_0) - u(0,X_0)\Big|^2 + \underset{0\leq t\leq y}{\sup}\Big|u(t,X_t) - u^N(t,X_t)\Big|^2 \\ + \frac{\lambda+1}{14}\ y\int_0^y \big|X_s^N- X_s\big|^2\ \di s + (\lambda+1)\ y\int_0^y \big|u^N(s,X_s) - u(s,X_s)\big|^2\ \di s \\+ \underset{0\leq t\leq y}{\sup}\bigg|\int_0^t \{\nabla u^N(s,X_s^N) - \nabla u^N(s,X_s)\}\ \di W_s\bigg|^2 +\\ \underset{0\leq t\leq y}{\sup} \bigg|\int_0^t \{\nabla u^N(s,X_s) - \nabla u(s,X_s)\}\ \di W_s\bigg|^2 \Bigg)
        \end{multline*}
        \begin{multline*}
        \leq 14\Bigg(\underset{0\leq t\leq T}{\sup}\Big|u^N(0,X_0) - u(0,X_0)\Big|^2 + \underset{0\leq t\leq T}{\sup}\Big|u(t,X_t) - u^N(t,X_t)\Big|^2 \\ + \frac{\lambda+1}{14}\ T\int_0^y \big|X_s^N- X_s\big|^2\ \di s + (\lambda+1)\ T\int_0^T \big|u^N(s,X_s) - u(s,X_s)\big|^2\ \di s \\+ \underset{0\leq t\leq y}{\sup}\bigg|\int_0^t \{\nabla u^N(s,X_s^N) - \nabla u^N(s,X_s)\}\ \di W_s\bigg|^2 +\\ \underset{0\leq t\leq T}{\sup} \bigg|\int_0^t \{\nabla u^N(s,X_s) - \nabla u(s,X_s)\}\ \di W_s\bigg|^2 \Bigg)
        \end{multline*}
        \begin{multline*}
        \leq 14\Bigg(\underset{0\leq t\leq T}{\sup}\Big|u^N(0,X_0) - u(0,X_0)\Big|^2 + \underset{0\leq t\leq T}{\sup}\Big|u(t,X_t) - u^N(t,X_t)\Big|^2 \\ + \frac{\lambda+1}{14}\ T\int_0^y \underset{0\leq t\leq s}{\sup}\big|X_s^N- X_s\big|^2\ \di s + (\lambda+1)\ T\int_0^T \big|u^N(s,X_s) - u(s,X_s)\big|^2\ \di s \\+\underset{0\leq t\leq y}{\sup}\bigg|\int_0^t \{\nabla u^N(s,X_s^N) - \nabla u^N(s,X_s)\}\ \di W_s\bigg|^2 +\\ \underset{0\leq t\leq T}{\sup} \bigg|\int_0^t \{\nabla u^N(s,X_s) - \nabla u(s,X_s)\}\ \di W_s\bigg|^2 \Bigg).
        \end{multline*}
        Taking the expectation, we obtain:
        \begin{multline*}
        w(y) \leq 14\Bigg(\underset{0\leq t\leq T}{\sup}\Big|u^N(0,X_0) - u(0,X_0)\Big|^2 + \E\bigg[\underset{0\leq t\leq T}{\sup}\Big|u(t,X_t) - u^N(t,X_t)\Big|^2\bigg]\\ + \frac{\lambda+1}{14}\ T\ \int_0^y w(s)\ \di s + (\lambda+1)\ T\ \E\bigg[\int_0^T \big|u^N(s,X_s) - u(s,X_s)\big|^2\ \di s\bigg]\\ +\E\bigg[ \underset{0\leq t\leq y}{\sup}\bigg|\int_0^t \{\nabla u^N(s,X_s^N) - \nabla u^N(s,X_s)\}\ \di W_s\bigg|^2 \bigg] +\\ \E\bigg[ \underset{0\leq t\leq T}{\sup}\bigg|\int_0^t \{\nabla u^N(s,X_s) - \nabla u(s,X_s)\}\ \di W_s\bigg|^2 \bigg]\Bigg).
        \end{multline*}
        
        Using Burkholder-Davis-Gundy inequality, it follows that:
        \begin{multline*}
        w(y) \leq 14\Bigg(\underset{0\leq t\leq T}{\sup}\Big|u^N(0,X_0) - u(0,X_0)\Big|^2 + \E\bigg[\underset{0\leq t\leq T}{\sup}\Big|u(t,X_t) - u^N(t,X_t)\Big|^2\bigg]\\ + \frac{\lambda+1}{14}\ T\ \int_0^y w(s)\ \di s + (\lambda+1)\ T\ \E\bigg[\int_0^T \big|u^N(s,X_s) - u(s,X_s)\big|^2\ \di s\bigg]\\ +2\int_0^y \E\bigg[\Big|\nabla u^N(s,X_s^N) - \nabla u^N(s,X_s)\Big|^2\bigg]\ \di s +2\int_0^T \E\bigg[\Big|\nabla u^N(s,X_s) - \nabla u(s,X_s)\Big|^2\bigg]\ \di s\Bigg)
        \end{multline*}
        
        \begin{multline*}
        \leq 14\Bigg(\underset{0\leq t\leq T}{\sup}\Big|u^N(0,X_0) - u(0,X_0)\Big|^2 + \E\bigg[\underset{0\leq t\leq T}{\sup}\Big|u(t,X_t) - u^N(t,X_t)\Big|^2\bigg]\\ + \frac{\lambda+1}{14}\ T\ \int_0^y w(s) \di s + (\lambda+1)\ T\ \E\bigg[\int_0^T \big|u^N(s,X_s) - u(s,X_s)\big|^2\ \di s\bigg]\\ +\frac{1}{7}\int_0^y\E\bigg[ \Big|X_s^N - X_s\Big|^{2\alpha}\bigg]\ \di s +2\int_0^T \E\bigg[\Big|\nabla u^N(s,X_s) - \nabla u(s,X_s)\Big|^2\bigg]\ \di s\Bigg)
        \end{multline*}
%        \begin{multline*}
%        \leq 12\Bigg(\underset{0\leq t\leq T}{\sup}\Big|u^N(0,X_0) - u(0,X_0)\Big|^2 + \E\bigg[\underset{0\leq t\leq T}{\sup}\Big|u(t,X_t) - u^N(t,X_t)\Big|^2\bigg]\\ + \frac{\lambda+1}{12}\ T\ \E\bigg[\int_0^T \underset{0\leq t\leq s}{\sup}\big|X_s^N- X_s\big|^2\ \di s\bigg]\\ + (\lambda+1)\ T\ \E\bigg[\int_0^T \big|u^N(s,X_s) - u(s,X_s)\big|^2\ \di s\bigg] +\frac{2}{3}\ T \Bigg)
%        \end{multline*}
        
        Applying fractional Morrey inequality, we obtain $\forall t\in[0,T]$:
        \begin{equation*}
        \begin{cases}
        \norme{u^N(t) - u(t)}_{L^\infty}\leq\norme{u^N(t) - u(t)}_{\mathcal{C}^{1,\alpha}}\leq c\norme{u^N(t)-u(t)}_{H^{-\beta}_{\tilde{q},q}}\\        
        \norme{\nabla u^N(t) - \nabla u(t)}_{L^\infty}\leq\norme{u^N(t) - u(t)}_{\mathcal{C}^{1,\alpha}}\leq c\norme{u^N(t)-u(t)}_{H^{-\beta}_{\tilde{q},q}}.
        \end{cases}        
        \end{equation*}
        
        Now, with
        \begin{equation*}
        \norme{u^N(t)-u(t)}_{H^{-\beta}_{\tilde{q},q}}\leq e^{\rho T} \norme{u^N(t)-u(t)}_{H^{-\beta}_{\tilde{q},q}}^{(\rho)}\leq Ke^T\norme{b^N-b}_{H^{-\beta}_{q}}
        \end{equation*} from lemma 22 in \cite{Fla-Iss-Rus-2017}, we have:
        
        \begin{multline*}
        w(y)
        \leq 14\Bigg(2c^2K^2e^{2\rho T}\norme{b^N-b}_{H^{-\beta}_{q}}^2 + \frac{\lambda+1}{14}\ T\ \int_0^y w(s)\ \di s\\+\frac{1}{7}\ \int_0^y\E\bigg[ \Big|X_s^N - X_s\Big|^{2\alpha}\bigg]\ \di s + (\lambda+1)\ T^2 c^2K^2e^{2\rho T}\norme{b^N-b}_{H^{-\beta}_{q}}^2\\ +2c^2K^2Te^{2\rho T}\norme{b^N-b}_{H^{-\beta}_{q}}^2 \Bigg)
        \end{multline*}
        
        \begin{multline*}
        \leq 14\left[2 + 2T + (\lambda+1)\ T^2\right] c^2K^2e^{2\rho T}\norme{b^N-b}_{H^{-\beta}_{q}}^2  +
        (\lambda+1)\ T\ \int_0^y w(s)\ \di s\\+2\ \int_0^y\E\bigg[ \underset{0\leq t\leq s}{\sup}\big|X_s^N - X_s\big|^{2\alpha}\bigg]\ \di s
        \end{multline*}        
        \begin{multline*}
        \leq 14\left[2 + 2T + (\lambda+1)\ T^2\right] c^2K^2e^{2\rho T}\norme{b^N-b}_{H^{-\beta}_{q}}^2  +
        (\lambda+1)\ T\ \int_0^y w(s)\ \di s\\+2\ \int_0^y\E\bigg[ \underset{0\leq t\leq s}{\sup}\big|X_s^N - X_s\big|^{2}\bigg]^\alpha\ \di s
        \end{multline*}
        \begin{multline*}
        \leq 14\left[2 + 2T + (\lambda+1)\ T^2\right] c^2K^2e^{2\rho T}\norme{b^N-b}_{H^{-\beta}_{q}}^2  +
        (\lambda+1)\ T\ \int_0^y w(s)\ \di s\\+2\ \int_0^y w(s)^\alpha\ \di s
        \end{multline*}
        
        
%        \textsc{Finally Gronwall inequality gives:
%        \begin{equation}
%        \E\left[\underset{0\leq t\leq T}{\sup}\left|X^{N}_t-X_t\right|^2\right]
%        \leq 
%        C\exp\left((\lambda+1)\ T^2\right)\norme{b^N-b}_{H^{-\beta}_{q}}^2        
%        \end{equation}
%        with $C=\left(14\left(2 + T + (\lambda+1)\ T^2\right) c^2K^2e^{2T}\right)$.}
       
\bibliographystyle{abbrv}
\bibliography{pre}
    
\end{document}